In the HoTT Book~\cite{hottbook},
the identity type is defined the same way as the one
in Martin-L\"{o}f Type Theory (hereafter as ``MLTT'')~\cite{MLTT},
but used differently (stated in Chapter 1 notes).
The elimination rule of the identity type is called \textbf{path induction},
but according to its definition we can tell
it's just another name of the MLTT \textbf J rule.

We distinguish them by calling the HoTT identity type the \textit{path type},
while the MLTT one as the \textit{identity type}.
The most notable difference is that the path type is
\textit{proof relevant} (implies the absence of
\textit{Axiom K}~\cite{AxiomK}).

By not providing a better definition of the path type,
we have to assume function extensionality as an axiom,
and we cannot compute transport on nontrivial paths.

To safely assume the univalence axiom, we need to avoid having Axiom K
in our type system.
Agda~\cite{Agda} implements~\cite{WithoutK} this via a flag \texttt{--without-K},
while in Coq~\cite{Coq}, Axiom K is not assumed.
Agda have further developed to have a without-K-compatible \textsf{Prop}
universe~\cite{PropWithoutK}, a without-K-compatible definitional equality
customization strategy~\cite{RewriteWithoutK}, and many other stuffs.
Another dependently-typed programming language similar to Agda,
Idris~\cite{Idris}, sticks to Axiom K and is
inconsistent~\cite{IdrisHoTT} with a postulated univalence axiom.

Even we have ways to avoid Axiom K,
we are still missing a constructive version of function extensionality,
and we also cannot compute the univalence axiom.

\subsection{Introduction}
\label{subsec:introduction}

This note is about giving HoTT a completely constructive interpretation,
from an implementation and user-experience perspective.
The categorical models behind are discussed only when necessary.

By giving a constructive interpretation of HoTT,
these issues are addressed:

\begin{itemize}
\item The path type should be constructive --
  there need to be formation, introduction and
  elimination rules (\cref{sec:path}).
\item Transport should compute (\cref{subsec:coe}).
\item The univalence axiom (\cref{sec:ua}) and function extensionality
  (\cref{subsec:path-prop}) should compute.
\item Inductive types~\cite{Inductive} should allow \textbf{path constructors}
  to form \textit{higher inductive types} (\cref{sec:hit}).
\end{itemize}

This note assumes the reader to:

\begin{itemize}
\item Have brief understanding of HoTT.
\item Have basic understanding of dependent type programming and MLTT.
\end{itemize}