\subsection{Interval types}
\label{subsec:interval}

The interval type has very simple formation rule
and introduction rule:

\[
  \vdash \mathbb I\ \textbf{type}
  \xtag \quad
  \vdash \textsf 0 : \mathbb I
  \xtag \quad
  \vdash \textsf 1 : \mathbb I
  \xtag
\]

The programming language Arend~\cite{Arend} uses a different notation
(\textsf{left} instead of \textsf 0, \textsf{right} instead of \textsf 1)
for interval endpoints.
We will still use \textsf 0 and \textsf 1 when talking
about Arend for consistency.

The interval type do not yet have an elimination rule,
so we cannot have a predicate on an interval.

By this definition of interval, the path type can
have the following introduction rule
(definitional equality between term $a$ and $b$
is denoted as $a \equiv b$,
usually implemented via conversion checking or normalization):

\[
  \cfrac{
    \Gamma \vdash a =_X b \ \textbf{type}
    \quad
    \Gamma, i : \mathbb I \vdash t : X
    \quad
    (\lambda i. t) \ \textsf 0 \equiv a
    \quad
    (\lambda i. t) \ \textsf 1 \equiv b
  }{
    \Gamma \vdash \lrangle i t : a =_X b
  }
  \xtag
\]

Heterogeneously:

\[
  \cfrac{
    \Gamma \vdash a =_F b \ \textbf{type}
    \quad
    \Gamma, i : \mathbb I \vdash t : F \ i
    \quad
    (\lambda i. t) \ \textsf 0 \equiv a
    \quad
    (\lambda i. t) \ \textsf 1 \equiv b
  }{
    \Gamma \vdash \lrangle i t : a =_F b
  }
  \xtag
\]

The above definition is used in Cubical Type Theory~\cite{CCHM,CHM}
(hereafter as CTT), Cartesian Cubical Type
Theory~\cite{CCTT,CCTT2,CHTT} (hereafter as CCTT).
There are three usable implementations of CTT described in~\cite{CHM},
including cubicaltt~\cite{CubicalTT},
mlang~\cite{Mlang} and Cubical Agda~\cite{CubicalAgda}.
For CCTT, there are two, including
redtt~\cite{RedTT} (somehow supersedes and deprecates
\RedPRL~\cite{RedPRL}) and yacctt~\cite{YaccTT},
which implement different variations of CCTT.

The syntax is intentionally made similar to a lambda abstraction,
as the introduction rule is the same as lambda abstraction with
additional two definitional equalities required.
The elimination rule for a path is therefore similar to an application,
with two additional reduction rules -- applying an interval $i$ to
an arbitrary term whose type is known to be a path type $a =_X b$
will reduce to $a$ if $i \equiv \textsf 0$ or $b$ if $i \equiv \textsf 1$:

\[
  \cfrac{
    \Gamma \vdash p : a =_X b
  }{
    \Gamma \vdash p\ \textsf 0 \equiv a
    \quad
    \Gamma \vdash p\ \textsf 1 \equiv b
  }
  \xtag \label{eqn:path-app}
\]

Rule~\ref{eqn:path-app} holds even if $p$ is a neutral term,
or a contructor (in case there's path constructors,
introduced in~\cref{subsec:path-hit}).
Therefore constructor application can be redex as well.

CTT and CCTT (and many other variations) have different primitive
operations defined for the interval type,
we discuss this later in~\cref{sec:kan}.

Arend on the other hand defines a primitive operator \textsf{path}
as the introduction rule for path:

\[
  \cfrac{
    \Gamma \vdash X \ \textbf{type}
    \quad
    \Gamma \vdash t : \mathbb I \rightarrow X
  }{
    \Gamma \vdash \textsf{path} \ t : (t \ \textsf 0) =_X (t \ \textsf 1)
  }
  \xtag
\]

The elimination of paths is still similar to CTT or CCTT.

CTT and CCTT do not support creating paths between intervals,
while Arend do.
Thus the following judgement holds in Arend, say,
that there exists a path between \textsf 0 and \textsf 1:

\[
  \vdash \textsf{path}\ (\lrangle i i) : \textsf 0 =_{\mathbb I} \textsf 1
  \xtag \label{eqn:0-1-arend}
\]

This does not mean that \textsf 0 is identical to \textsf 1.
They are not equivalent definitionally, but propositionally.
Here's the concrete syntax of~\ref{eqn:0-1-arend} in Arend:

\begin{minted}{arend-lexer.py:ArendLexer -x}
  \func interval-path : left = right => path (\lam i => i)
\end{minted}

\subsubsection{Interval operations}

Different model defines different operations on the $\mathbb I$ type.
Commonly defined operations are all subsets of de Morgan algebraic operations,
making $\mathbb I$ a bounded distributive lattice.

The three operations (min -- $\wedge$, max -- $\vee$, and involution $\neg$)
extended to the fundamental definition are of the following typing rules:

\[
  \cfrac{
    \Gamma \vdash i : \mathbb I
    \quad
    \Gamma \vdash j : \mathbb I
  }{
    \Gamma \vdash i \wedge j : \mathbb I
    \quad
    \Gamma \vdash i \vee j : \mathbb I
  }
  \xtag
  \quad
  \cfrac{
    \Gamma \vdash i : \mathbb I
  }{
    \Gamma \vdash \neg i : \mathbb I
  }
  \xtag
\]

The interval themselves satisfy the following definitional equalities,
given $r : \mathbb I, s : \mathbb I$:

\[
  \neg \textsf 0 \equiv \textsf 1 \xtag \quad
  \neg \textsf 1 \equiv \textsf 0 \xtag
\]
\[
  \neg (r \vee s) \equiv (\neg r) \wedge (\neg s) \xtag
\]
\[
  \neg (r \wedge s) \equiv (\neg r) \vee (\neg s) \xtag
\]

The involution operation really inverts an interval, so it proves symmetry:

\[
  \cfrac{
    \Gamma \vdash p : a =_A b
  }{
    \Gamma \vdash (\lrangle i\ p (\neg i)) : b =_A a
  } \xtag
\]

We propose that instead of saying it \textit{proves} symmetry,
it's better to say that symmetry is \textit{defined} by involution.

The other two operations, $\wedge$ and $\vee$
(commonly referred to as \textit{connections}),
are relevant to fillers~\cref{subsec:fill}.
We talk about them then.

CTT has these three operations, while CCTT have non of these.
Arend have non of these either.
Another variation of CTT,
the OP model~\cite{OPModel}, defines the connections only.

