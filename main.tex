\documentclass{article}

\usepackage[utf8]{inputenc}
\usepackage{url}
\usepackage{amsmath}
\usepackage{amsfonts}
\usepackage{etoolbox}
\usepackage{amsthm}
\usepackage[usenames,dvipsnames]{xcolor}
\usepackage[pagebackref,citecolor=blue,linkcolor=OliveGreen,urlcolor=Mahogany,colorlinks]{hyperref}
\usepackage{color}
\usepackage{cleveref}
\usepackage{tikz}
\usetikzlibrary{positioning}

\crefformat{section}{\S#2#1#3} % see manual of cleveref, section 8.2.1
\crefformat{subsection}{\S#2#1#3}
\crefformat{subsubsection}{\S#2#1#3}

\newcommand{\TODO}[0]{{\color{red} TODO}}
\newcommand{\lrangle}[1]{\langle #1\rangle}

\title{Constructive Interpretations of HoTT}
\author{Tesla Ice Zhang}

% allow multiple labels in displaymath
\makeatletter
\patchcmd{\mathdisplay}
{\let\label\label@in@display}{}
{}{\fail}
\makeatother

% new cmd
\newcommand\xtag{
\refstepcounter{equation}%
\;\;{\color{orange} \text{(\theequation)}}}
\newcommand{\refl}{\textsf{refl}}

\begin{document}
\maketitle

\tableofcontents

\section{Introduction}

In the HoTT Book~\cite{hottbook},
the identity type is defined the same way as the one
in Martin-L\"{o}f Type Theory (hereafter as ``MLTT'')~\cite{MLTT},
but used differently (stated in Chapter 1 notes).
The elimination rule of the identity type is called \textbf{path induction},
but according to its definition we can tell
it's just another name of the MLTT \textbf J rule.

We distinguish them by calling the HoTT identity type the \textit{path type},
while the MLTT one as the \textit{identity type}.
The most notable difference is that the path type is
\textit{proof relevant} (implies the absence of \textit{Axiom K}).

By not providing a better definition of the path type,
we have to assume function extensionality as an axiom,
and we cannot compute transport on nontrivial paths.

Although there are ways to avoid Axiom K
(so we can safely assume the univalence axiom)
somehow with the identity type~\cite{WithoutK},
we are still missing a constructive version of function extensionality,
and we also cannot compute the univalence axiom.

\bibliography{ref}
\bibliographystyle{apalike}

\end{document}
