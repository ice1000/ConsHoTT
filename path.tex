To define Path constructively,
we may get some inspiration from its topological definition.

Open-up the \href{https://ncatlab.org/nlab/show/path}{``path'' segment in nLab},
there is a mathematical definition of a path, written as:

\[
  \mathbb I \rightarrow X
  \xtag
\]

Topologically, a path in a space $X$ is a continuous map
from an interval (denoted $\mathbb I$) to $X$.
Type theoretically, $\mathbb I$ and $X$ are supposed to be types,
while $\rightarrow$ forms function types.
As paths represent a relation \textit{between} two terms
(endpoints, but type theoretically),
these two terms should show up in the path type as well
(similar to MLTT identity type).
Therefore the formation rule for path types is naturally:

\[
  \cfrac{
    \Gamma \vdash X \ \textbf{type}
    \quad
    \Gamma \vdash a : X
    \quad
    \Gamma \vdash b : X
  }{\Gamma \vdash a =_X b \ \textbf{type}}
  \xtag
\]

Up to the time where this note is written,
everyone tries to define constructive HoTT defines path types this way.
They have different introduction and elimination rules,
but all of their introduction rules are
based on the interval type $\mathbb I$.

We can also define heterogeneous path types
(path between two terms of different types)
by changing the type $X$ into a type family $f$,
indexed by the interval type $\mathbb I$:

\[
  \cfrac{
    \Gamma \vdash X \ \textbf{type}
    \quad
    \Gamma \vdash a : X
    \quad
    \Gamma \vdash b : X
    \quad
    \Gamma \vdash f : \mathbb I \rightarrow X
  }{\Gamma \vdash a =_f b \ \textbf{type}}
  \xtag
\]

\subsection{Interval types}

The interval type has very simple formation rule
and introduction rule:

\[
  \vdash \mathbb I\ \textbf{type}
  \xtag \quad
  \vdash \textsf 0 : \mathbb I
  \xtag \quad
  \vdash \textsf 1 : \mathbb I
  \xtag
\]

The programming language Arend~\cite{Arend} uses a different notation
(\textsf{left} instead of \textsf 0, \textsf{right} instead of \textsf 1)
for interval endpoints.
We will still use \textsf 0 and \textsf 1 when talking
about Arend for consistency.

The interval type do not yet have an elimination rule,
so we cannot have a predicate on an interval.

By this definition of interval, the path type can
have the following introduction rule
(definitional equality between term $a$ and $b$
is denoted as $a \equiv b$,
usually implemented via conversion checking or normalization):

\[
  \cfrac{
    \Gamma \vdash a =_X b \ \textbf{type}
    \quad
    \Gamma, i : \mathbb I \vdash t : X
    \quad
    (\lambda i. t) \ \textsf 0 \equiv a
    \quad
    (\lambda i. t) \ \textsf 1 \equiv b
  }{
    \Gamma \vdash \lrangle i t : a =_X b
  }
  \xtag
\]

Heterogeneously:

\[
  \cfrac{
    \Gamma \vdash a =_F b \ \textbf{type}
    \quad
    \Gamma, i : \mathbb I \vdash t : F \ i
    \quad
    (\lambda i. t) \ \textsf 0 \equiv a
    \quad
    (\lambda i. t) \ \textsf 1 \equiv b
  }{
    \Gamma \vdash \lrangle i t : a =_F b
  }
  \xtag
\]

The above definition is used in Cubical Type Theory~\cite{CCHM,CHM}
(hereafter as CTT), Cartesian Cubical Type
Theory~\cite{CCTT,CCTT2,CHTT} (hereafter as CCTT).
There are three usable implementations of CTT described in~\cite{CHM},
including cubicaltt~\cite{CubicalTT},
mlang~\cite{Mlang} and Cubical Agda~\cite{CubicalAgda}.
For CCTT, there are two, including
redtt~\cite{RedTT} and yacctt~\cite{YaccTT},
implementing different models.

The syntax is intentionally made similar to a lambda abstraction,
as the introduction rule is the same as lambda abstraction with
additional two definitional equalities required.
The elimination rule for a path is therefore similar to an application,
with two additional reduction rules -- applying an interval $i$ to
an arbitrary term whose type is known to be a path type $a =_X b$
will reduce to $a$ if $i \equiv \textsf 0$ or $b$ if $i \equiv \textsf 1$:

\[
  \cfrac{
    \Gamma \vdash p : a =_X b
  }{
    \Gamma \vdash p\ \textsf 0 \equiv a
    \quad
    \Gamma \vdash p\ \textsf 1 \equiv b
  }
  \xtag \label{eqn:path-app}
\]

Rule~\ref{eqn:path-app} holds even if $p$ is a neutral term,
or a contructor (in case there's path constructors,
introduced in~\cref{sec:hit}).
Therefore constructor application can be redex as well.

CTT and CCTT (and many other variations) have different primitive
operations defined for the interval type,
we discuss this later in~\cref{sec:kan}.

Arend on the other hand defines a primitive operator \textsf{path}
as the introduction rule for path:

\[
  \cfrac{
    \Gamma \vdash X \ \textbf{type}
    \quad
    \Gamma \vdash t : \mathbb I \rightarrow X
  }{
    \Gamma \vdash \textsf{path} \ t : (t \ \textsf 0) =_X (t \ \textsf 1)
  }
  \xtag
\]

The elimination of paths is still similar to CTT or CCTT.

\subsection{Path properties}
\label{subsec:path-prop}
\input{latex/path-properties}

\subsection{Transport}
\label{subsec:coe}

The idea of \textit{type-safe coercion} is like
casting a variable of type $A$ to type $B$ by providing a proof
of $A =_{\mathcal U} B$:

\[
  \textsf{coe} : A =_{\mathcal U} B \rightarrow A \rightarrow B
  \xtag \label{eqn:coe-a-b-id}
\]

Bringing in the idea of path types,
assuming $p : A =_{\mathcal U} B$, by~\ref{eqn:path-app}
we know that $p \ \textsf 0 \equiv A$ and
$p \ \textsf 1 \equiv B$,
so~\ref{eqn:coe-a-b-path} is effectively the same
as~\ref{eqn:coe-a-b-id}:

\[
  \textsf{coe} : (p : A =_{\mathcal U} B) \rightarrow p \ \textsf 0
  \rightarrow p \ \textsf 1
  \xtag \label{eqn:coe-a-b-path}
\]

The return type of \textsf{coe} can be generalized over
the interval on the input path:

\[
  \textsf{coe} : (p : A =_{\mathcal U} B) \rightarrow p \ \textsf 0
  \rightarrow (i : \mathbb I) \rightarrow p \ i
  \xtag \label{eqn:coe-a-b-path-gen}
\]

\ref{eqn:coe-a-b-path-gen} is known as the ``generalized transport''
operation, denoted as \textsf{coe} in Arend and CCTT,
and \AgdaFunction{transp} in CTT.
We use \textsf{coe} to refer to this general concept for consistency.
And by the way,
Arend's \textsf{coe} takes function over interval instead
of a path as its first argument:

\[
  \textsf{coe} : (p : I \rightarrow \mathcal U)
  \rightarrow p \ \textsf 0
  \rightarrow (i : \mathbb I) \rightarrow p \ i
  \xtag
\]

Right now there is neither nontrivial path nor path on types,
while transport along identity paths is just a fancy version
of the identity function, thus we do not need to worry about the
computation of \textsf{coe} yet.
We further discuss \textsf{coe} in~\cref{sec:ua}.

In CTT and CCTT, we cannot transport along paths between paths,
but it's not the case in Arend.
Arend uses \textsf{coe} to prove path symmetry and transitivity.
Here's a proof, assuming three inhabitants $a, b, c$ of type $A$
and the trivial path $\refl_a : a =_A a$:

\[
  \cfrac{
    \Gamma \vdash p : a =_A b
  }{
    \Gamma \vdash (\textsf{coe}
    (\lambda i. (p\ i =_A a)) \ \refl_a \ \textsf 1)
    : b =_A a
  }
  \xtag \label{eqn:arend-proof-sym}
\]

Note that in Arend's concrete syntax, path application is not
$p\ i$ but $p\ @\ i$ and lambda abstraction is not $\lambda x.y$ but
$\backslash{}\textsf{lam} \ x \Rightarrow y$.
What happened in~\ref{eqn:arend-proof-sym} is that we transported
along the following interval lambda:

\[
  f = \lambda i. (p\ i =_A a)
  \xtag
\]

As $p : a =_A b$, we have
$(f \ \textsf 0) \equiv (p \ \textsf 0 =_A a) \equiv (a =_A a)$ and
$(f \ \textsf 1) \equiv (p \ \textsf 1 =_A a) \equiv (b =_A a)$,
so $f$ is similar to a path $(a =_A a) =_{\mathcal U} (b =_A a)$.
By providing a proof \refl of $a =_A a$,
we obtain the proof of the right hand side of $f$, which is $b =_A a$.
The proof of transitivity is similar, as in~\ref{eqn:arend-proof-trans}.
Analyzing how it works is left as an exercise.

\[
  \cfrac{
    \Gamma \vdash p : a =_A b
    \quad
    \Gamma \vdash q : b =_A c
  }{
    \Gamma \vdash (\textsf{coe}
    (\lambda i. (a =_A q\ i)) \ p \ \textsf 1)
    : a =_A c
  }
  \xtag \label{eqn:arend-proof-trans}
\]

\subsection{Homotopies}
\label{subsec:hom}

\TODO
