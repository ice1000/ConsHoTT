By the above constructive path type,
we can extend inductive types with path constructors.
Recall that a constructor of an inductive type $T$ is
similar to a function whose return type is $T$,
but does not reduce.

Inductive types with path constructors are called
\textit{higher inductive types} (hereafter as HIT),
which is discussed in Chapter 6 of the HoTT Book.
Here's a simple example of HIT,
where the name \textsf{Seg} stands for \textit{Segment}:

\[
  \vdash \textsf{Seg} \ \textbf{type}
  \xtag
\]
\[
  \vdash \textsf{left} : \textsf{Seg}
  \xtag
  \quad
  \vdash \textsf{right} : \textsf{Seg}
  \xtag
\]
\[
  \vdash \textsf{inner} :
  \textsf{left} =_{\textsf{Seg}} \textsf{right}
  \xtag
\]

The type \textsf{Seg} has two \textbf{point constructors}
\textsf{left} and \textsf{right}, with one \textbf{path constructor}
\textsf{inner} whose two endpoints are \textsf{left} and \textsf{right}.

Path constructors constraint the operations defined on HITs.
Taking the \textsf{Seg} type as an example,
assume arbitrary type $A$, function $f : \textsf{Seg} \rightarrow A$
should satisfy that the two endpoints of the path $p = \textsf{ap}_f(\textsf{inner})$
should definitionally equal (notationally, $p\ \textsf 0 \equiv p\ \textsf 1$).

Generally, given arbitrary HIT $T$ and arbitrary type $A$,
function $f : T \rightarrow A$ should satisfy that for all path constructors
$p : a =_T b$ of $T$, $f\ a \equiv f\ b$ holds definitionally.
We call this property ``agree on path constructors'',
or ``respect the path constructors''.

Higher inductive types are similar in CTT and CCTT,
while Arend is somehow different from the other two.

\subsection{Axiomatic approach}

In the old days, there were no constructive path type.
People use the MLTT identity type and
work with HITs by postulating their existence.
What they're actually postulating is the existence of the path constructors.

There had been a lot of work done based on this approach,
under various proof assistants such as Hoq
(a modified version of Coq, short for HoTT-Coq),
% TODO: is lean based on this?
and Agda (before cubical).

Many HITs can be generalized as a specialized version of a standard
quotient type. If we only postulate a quotient type, we can specialize
a number of HITs into the quotient to avoid postulating things again.

In the HoTT library~\cite{HottCoq} of Coq, the comments in the quotient
type module provide an ideal syntax for the quotient type definition:

\begin{minted}[fontsize=\small]{coq}
Inductive quotient : Type :=
| class_of : A -> quotient
| related_classes_eq : forall x y, (R x y), class_of x = class_of y
| quotient_set : IsHSet quotient.
\end{minted}

The \href{https://github.com/HoTT/HoTT/blob/b20bb573739284349a968bb219405255c744049d/theories/HIT/quotient.v#L40-L42}
{actual definition} is:

\begin{minted}[fontsize=\small]{coq}
Private Inductive quotient {sR: is_mere_relation _ R} : Type :=
| class_of : A -> quotient.
Axiom quotient_set : IsHSet (@quotient sR).
\end{minted}

And, of course, there's no way to ensure that functions defined on
axiomatic HITs agree on their path constructors.

\input{latex/hit-agda-old}

\subsection{Conditions}

Arend has built-in support for HITs
similar to Agda's old approach, say, provide
path constructors as something similar to a rewriting rule, namely \textit{conditions}.
However, Arend requires operations defined on HITs to agree on the path constructors,
thus it's more consistent to the theoretical model of HITs in HoTT.

\TODO

\subsection{Path as constructor}

Here's the solution in CTT and CCTT.

\TODO